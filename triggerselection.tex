\section{Trigger Selection}
\label{sec:triggerselection}
As discussed in Section \ref{sec:triggersystem} the majority of data coming from the detector is discarded. 
For this reason it is important that an appropriate trigger path exists for any analysis.
For this analysis a number of muon + hadronic tau cross triggers are used to filter the data coming from the detector.
The cross trigger uses a combination of an isolated muon trigger along with an isolated tau trigger.
Where possible the trigger used is the lowest possible transverse momentum ($p_{T}$) combination that remains unprescaled.
For this reason the trigger $p_{T}$ thresholds used change for different run ranges and scale with the luminosity being delivered to the CMS detector.
Towards the end of the 2011 data taking an additional requirement on the muon pseudo-rapidity was added.
The primary driving force in the choice of $p_{T}$ cuts outlined in Section \ref{sec:particleselection} is the rising edge of the trigger efficiency close to the $p_{T}$ threshold of the trigger object, the treatment of the trigger efficiency will be discussed in Chapter \ref{chap:systematics}.
The specific triggers used in the analysis consist of an isolated muon, in most cases with $p_{T} > 15$ GeV, combined with a loosely isolated particle flow tau object with $p_{T} > 10$ GeV, $p_{T} > 15$ GeV, or $p_{T} > 20$ GeV, depending on the luminosity.

\begin{table}[tpb]
  \setlength{\capwidth}{0.9\textwidth}
  \begin{small}
  \begin{center}
    \caption{HLT TRIGGER PATHS USED}
    \label{tab:triggerpaths}
    \begin{tabular}{lcc}
      \toprule
      HLT Trigger Path & Run Range & Int. Luminosity [$fb^{-1}$] \\
      \midrule
      HLT\_IsoMu12\_LooseIsoPFTau10 		& 160431-163869 & 0.017 \\
      HLT\_Mu15\_LooseIsoPFTau20    		& 160431-163869 & 0.017 \\
      HLT\_IsoMu15\_LooseIsoPFTau15 		& 165088-178420 & 1.97 \\  
      HLT\_IsoMu15\_eta2p1\_LooseIsoPFTau20 	& 173236-180252 & 2.46 \\
      \bottomrule
    \end{tabular}
  \end{center}
  \end{small}
\end{table}

