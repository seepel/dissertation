In order to reject events arising from $Z \rightarrow \mu^{-}\mu^{+}$ an event is rejected based on the existence of a second loosely selected muon.
The second loosely selected muon is defined as a muon that has a track in the inner tracker matched to a track in the muon system, a  $p_{T} > 15$ GeV, $-2.4 < \eta < 2.4$, and $I_{rel} < 0.15$.
Once a collection of loosely selected muons is compiled the loosely selected muons are matched to muons selected as in Section \ref{sec:muonselection} using a maximum distance ($\Delta R$) in $\eta-\phi$ space of $0.5$.
An event is rejected if there are any matched muon pairs found in the event.
A similar procedure is repeated again in order to reduce backgrounds arising from the production of $Z/\gamma^{*} \rightarrow \mu^{-}\mu^{+}$.
This second di-muon rejection is performed with muons that satisfy the previous selection with the exception that the second muon does not have the $p_{T}$ and $I_{rel}$ requirements, additionally the di-muon pair is required to have zero charge and match within $\Delta R < 1.0$.
