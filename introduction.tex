\chapter{INTRODUCTION}
The Large Hadron Collider (LHC) was built to study particle physics with proton-proton collisions at a high center-of-mass energy.
%Of particular interest is physics above the TeV scale.
The standard model (SM) is the current model of particle physics and has had great success in the prediction of physical observations.
Despite the great success of the SM there is one outstanding piece that is missing, a particle that has not yet been observed called the Higgs boson.
In addition to the missing Higgs boson, the SM also contains many hints that there exist new physical processes that will be exhibited at or around the TeV energy scale.
Such missing pieces could include new physics models such as supersymmetry, extra dimensions, or some yet unknown process or processes.
The Compact Muon Solenoid (CMS) detector is a general purpose detector located at one of the four collision points in the LHC.
The CMS detector is designed to explore new physics models that may lie at the TeV energy scale, as well as to definitively observe the Higgs boson and its properties.

This thesis presents a search for the Higgs boson as predicted by the SM, as well as neutral supersymmetric Higgs bosons.
The analysis presented is a search for a neutral Higgs boson decaying into two tau leptons, in which one tau lepton decays to a muon, and the other decays hadronically.
The analysis is performed on the CMS dataset collected during the 2011 physics run representing an integrated luminosity of $4.6$ fb$^{-1}$ at $\sqrt{s} = 7$ TeV.

At the time of this writing the Higgs boson has not yet been discovered, however both the CMS detector and the ATLAS detector at the LHC have found an excess in data that may indicate that the discovery of the Higgs boson is around the corner.
The search for the Higgs has been broken up into separate analyses that each target a different set of decay products which are then combined together to provide a greater sensitivity to the measurement of the Higgs boson.
If the mass of the Higgs boson is below $150$ GeV, the channel that provides the greatest sensitivity is that in which the Higgs boson decays to two photons.
In the case that the Higgs boson is heavier, the channels of most interest are those in which the Higgs boson decays into two vector bosons ($W^{\pm}$/$Z^{0}$). 
Although the previously mentioned channels are the most important in measuring the Higgs boson specified by the standard model, the case where the Higgs boson decays into two tau ($\tau$) leptons is less sensitive but still relevant.
The decay of the Higgs boson into two tau leptons, $H\rightarrow\tau\tau$, becomes more relevant in the case that new physics exists that modifies the Higgs sector.
In particular it has been shown that the supersymmetric Higgs sector may couple more strongly to the tau lepton making this channel an excellent probe of supersymmetry.

In Chapter \ref{chap:theory} the search for the Higgs boson will be motivated by the description of the SM, and in particular the Higgs mechanism which is a favored method by which the electromagnetic and weak interactions can be unified.
This chapter will then conclude with a summary of the Higgs search to date, as well as a brief description of the properties of the tau lepton.
Chapter \ref{chap:detector} will describe the experimental apparatus used for the analysis including a brief description of the LHC and the CMS detector with an emphasis on the muon system as it is this system that gives the analysis its discriminating power.
Physics simulations play an important role in many physics analyses, Chapter \ref{chap:detector} will also provide an introduction to the simulations used in this particular analysis.
In Chapter \ref{chap:analysis} the analysis itself will be discussed, giving a full description of the steps used in selecting the Higgs boson candidate events.
There are several background processes other than the decay of a Higgs boson that can produce the event signature of a muon and a tau lepton, these backgrounds will be presented in Chapter \ref{chap:backgrounds}.
In addition to simply describing the backgrounds, Chapter \ref{chap:backgrounds} will also present a method by which the size and distribution of the background processes can be estimated.
Chapter \ref{chap:systematics} will cover certain corrections that must be made to the simulated datasets used in the analysis, as well as discuss and estimate the systematic uncertainties that are associated with the measurement.
Finally, in Chapter \ref{chap:results} the results of the analysis will be presented, giving both the selected event yields and the limits that the analysis sets for the production rate and subsequent decay of the Higgs boson in both the SM and the supersymmetric model considered.
