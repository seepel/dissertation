\section{Superconducting Magnet}
\label{sec:magnet}
As the name the Compact Muon Solenoid implies the superconducting solenoid magnet is a central feature of the CMS detector.
The magnetic field is necessary to measure the momentum of charged particles produced in the proton-proton collisions.
As a charged particle travels through a magnetic field the radius of curvature for the charged particle is given by
\begin{equation}
r = {p_{\perp}\over|q|B},
\end{equation}
where $p_{\perp}$ is the component of the particle's momentum that is transverse to the direction of the magnetic field, $q$ is the charge of the particle, and B is the strength of the magnetic field.
As the radius of curvature is proportional to $p_{\perp}$, it becomes clear that to measure high momentum particles a very large magnetic field is required.
In addition to the requirement for the magnetic field to be large, it is also important that magnetic field be homogeneous in the volume of the detector to reduce systematic errors that result from a non-uniform field.

In order to provide a homogeneous magnetic field the CMS solenoid is required to be physically large, encapsulating the tracking system and calorimeters.
It has a radial bore of 6.3 meters, a length of 12.9 meters, and a weight of 220 tons.
The solenoid is made of four layers of wire with a total of 2168 turns carrying a nominal current of 19.14 kA.
A second momentum measurement is applied to muons that are detected in the muon system surrounding the solenoid, for this reason the muon system is interspersed throughout the iron return yoke.
The return yoke has the effect of minimizing the fringe field outside the solenoid, providing a homogeneous field throughout the muon system. %FIXME return yoke necessary
