\subsection{Quantum Chromodynamics}
\label{sec:qcd}

% gluon -> charge
% no colored object
% asymptotic freedom
Quantum Chromodynamics (QCD)  is the theory that governs the strong nuclear interactions and thus describes the interactions of quarks and gluons. 
Similar to the theories previously discussed, QCD is a quantum field theory, except that the symmetry is $SU(3)$.
The consequence of the symmetry group leads to a total of three color charges which are defined as red, green and blue.
Unlike the theory of QED, QCD is a non-Abelian gauge theory, which has the effect that gluons carry a color charge. 
Since the gluons carry a color charge, this means that a gluon can interact with other gluons.
QCD is invariant under color transformations yielding the consequence that composite objects (i.e. any observable particle) must carry no net color charge.
Quarks as well as gluons carry a color charge, thus bare quarks can not exist alone, leading to the concept of color confinement.
Color confinement can be described by examining an attempt to pull two quarks apart from each other. 
As the distance between the quarks increases, the energy of the system also increases.
As the energy increases it eventually becomes energetically favorable to create two additional quarks from the vacuum which then bind to the original quarks in question, creating two color neutral hadrons.

The QCD Lagrangian is given by:
\begin{equation}
\Lagr_{QCD} = -\frac{1}{4}F_{\mu\nu}^{a}F^{\mu\nu a} + i\sum_{q}\psibar_{q}^{i}\gamma^{\mu}(D_{\mu})_{ij}\psi_{q}^{j} - \sum_{q}m_{q}\psibar_{q}^{i}\psi_{qi}, 
\end{equation}
where $\psi$ is the 4-component Dirac spinor for the quark field, and are summed over the colors $i$ and the flavors $q$.
By examining this Lagrangian, the points made in the preceding paragraph are enumerated.
The first term represents the kinetic energy of the gluon, similar to that for the photon in Equation \ref{eqn:qedlagrangian}.
The second term represents the interaction between quarks and gluons as well as the kinetic energy of the quarks, while the third term represents the mass of the quarks.
Although similar to the kinetic term of QED, due to the $SU(3)$ symmetry group of QCD, the free field QCD term yields eight gluon fields.
The free field QCD term is:
\begin{equation}
F_{\mu\nu}^{a} = \partial_{\mu}A_{\nu}^{a} - \partial_{\nu}A_{\mu}^{a} - g_{s}f_{abc}A_{\mu}^{b}A_{\nu}^{c},
\label{eqn:qcdfreefield}
\end{equation}
where $f_{abc}$ are the structure constants of the $SU(3)$ algebra, playing a similar role to the Pauli spin matrices.
In Equation \ref{eqn:qcdfreefield}, $A_{\mu}^{a}$ are the gluon fields, and $g_{s}$ is the strong coupling constant.
The third term in Equation \ref{eqn:qcdfreefield} demonstrates the self interaction of the gluon fields.

For the purposes of this analysis, the primary consequence of the QCD theory is that of color confinement.
In practice, when a quark is produced in a high-energy collision, as it travels away from the interaction it will hadronize.
This hadronization process will result in a spray of hadrons that are produced surrounding the initial quark.
In executing an experimental high energy physics analysis, one does not observe quarks directly, but rather observes a ``jet'' of hadrons that result from the hadronization process.
The jet will contain numerous charged and neutral hadrons, primarily pions. 
%Heavier hadrons however, can arise and often lead to jet sub-structure.
Finally, QCD processes typically have cross sections many orders of magnitude larger than other processes, and thus can contribute large unwanted backgrounds to an analysis. 


%\begin{figure}
%\begin{center}
%\begin{fmffile}{firstorderqcd} 	%one.mf will be created for this feynman diagram  
%\fmfframe(1,7)(1,7){ 	%Sets dimension of Diagram
%\begin{fmfgraph*}(110,62) %Sets size of Diagram
%\fmfleft{i2,i1}	%Sets there to be 2 sources 
%\fmfright{o2,o1}    %Sets there to be 2  outputs
%\fmflabel{$u$}{i1} %Labels one of the left sources
%\fmflabel{$d$}{i2} %Labels one of the left sources
%\fmflabel{$d$}{o2} %Labels one of the right outputs
%\fmflabel{$u$}{o1} %Labels one of the right outputs
%\fmf{fermion}{i1,v1,o1} %Connects the sources with a vertex.
%\fmf{fermion}{i2,v2,o2} %Connects the sources with a vertex.
%\fmf{gluon}{v1,v2}
%\end{fmfgraph*}
%}
%\end{fmffile}
%\end{center}
%\end{figure}

%\begin{figure}
%\begin{center}
%\begin{fmffile}{higherorderqcd} 	%one.mf will be created for this feynman diagram  
%\fmfframe(1,7)(1,7){ 	%Sets dimension of Diagram
%\begin{fmfgraph*}(110,62) %Sets size of Diagram
%\fmfleft{id,iu}
%\fmf{fermion}{id,vdg1,vdg2,vdg3,vdg4,od}
%\fmf{fermion}{iu,vug1,vug2,ou}
%\fmf{gluon,right=.5,tension=.1}{vdg1,vdg4}
%\fmf{gluon,tension=.1}{vdg2,vug1}
%\fmf{gluon,tension=.1}{vdg3,vug2}
%\fmfright{od,ou}
%\end{fmfgraph*}
%}
%\end{fmffile}
%\end{center}
%\caption{QCD}
%\end{figure}


