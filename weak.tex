\subsection{Weak Interaction}
\label{sec:weak}
The theory of weak interactions was developed to explain the phenomenon of radioactive beta decay.
Enrico Fermi first developed the theory to explain neutron beta decay, or the transformation of a neutron into a proton with the emission of an electron  and a neutrino.
Fermi's construction of the theory was originally a contact theory, meaning that all fermions shared a common intersection vertex as can be seen in Figure \ref{fig:betadecaycontact}.
The Hamiltonian of Fermi's theory for beta decay is given as:
\begin{equation}
\label{eqn:fermibetadecay}
H = \frac{G_{\beta}}{\sqrt{2}}\left[\psibar_{p}\gamma_{\mu}(1- g_{A}\gammafive)\psi_{n}\right]\left[\psibar_{e}\gamma^{\mu}(1-\gammafive)\psi_{\nu}\right] + h.c.,
\end{equation}
where $G_{\beta}$ is the Fermi constant, $g_{A}$ is the relative fraction of the interaction, or coupling constant, and must be measured experimentally.
In Equation \ref{eqn:fermibetadecay}, the inclusion of the helicity operator $(1-\gammafive)$ gives rise to the observation that the theory violates parity. 
The parity violation of $(1-\gammafive)$  can be seen in the following relations between it and the left and right handed helicity states:
\begin{equation}
\begin{split}
h = (1-\gammafive)/2, \\
h\psi_{R} = \onehalf\psi_{R}, \\
h\psi_{L} = -\onehalf\psi_{L}. 
\end{split}
\end{equation}
Following the Hamiltonian presented in Equation \ref{eqn:fermibetadecay}, the amplitude for nuclear beta decay is given by:
\begin{equation}
M = \frac{G_{F}}{\sqrt{2}}\left[\overline{u}_{p}\gamma_{\rho}\frac{1-\gammafive}{2}u_{n}\right] \left[\overline{u}_{\nu_{e}}\gamma^{\rho}\frac{1-\gammafive}{2}u_{e}\right].
\end{equation}
The presence of $(1-\gammafive)$ essentially functions to eliminate terms with right handed helicity states.
\begin{figure}[htpb]
\begin{center}
\begin{fmffile}{betadecaycontact} 	%one.mf will be created for this feynman diagram  
\fmfframe(1,7)(1,7){ 	%Sets dimension of Diagram
\begin{fmfgraph*}(110,62) %Sets size of Diagram
\fmfleft{i1}	%Sets there to be 2 sources 
\fmfright{i2,o2,o1}    %Sets there to be 2  outputs
\fmflabel{$n$}{i1} %Labels one of the left sources
\fmflabel{$\nu$}{i2} %Labels one of the left sources
\fmflabel{$p$}{o2} %Labels one of the right outputs
\fmflabel{$e^{-}$}{o1} %Labels one of the right outputs
\fmf{fermion}{i1,v1,i2} %Connects the sources with a vertex.
\fmf{fermion}{i1,v1,o1} %Connects the sources with a vertex.
\fmf{fermion}{i1,v1,o2} %Connects the sources with a vertex.
\end{fmfgraph*}
}
\end{fmffile}
\caption{Feynman diagram of neutron beta decay in Fermi's contact theory.}
\label{fig:betadecaycontact}
\end{center}
\end{figure}

Although quite successful in describing nuclear beta decay, Fermi's contact theory fails in describing scattering processes at higher energies.
When applied in this scenario the cross section grows with energy. 
The result is that for some energy, the probability for the interaction becomes greater than one, violating unitarity.
A solution to this was developed by replacing the contact interaction with a massive intermediate gauge boson as can be seen in Figure \ref{fig:betadecay}.
The amplitude for the process under this regime becomes:
\begin{equation}
M = - \left[\frac{g}{\sqrt{2}}\overline{u}_{u}\gamma_{\rho}\frac{1-\gammafive}{2}u_{d}\right] \frac{-g^{\rho\sigma} + \frac{q^{\rho}q^{\sigma}}{M_{W}^{2}}}{q^{2} - M_{W}^{2}}\left[\frac{g}{\sqrt{2}}\overline{u}_{\nu_{e}}\gamma_{\rho}\frac{1-\gammafive}{2}u_{e}\right].
\end{equation}
The propagator contains the term $\frac{1}{q^{2} - M_{W}^{2}}$ which can be shown to converge to the contact theory at the low energy limit:
\begin{equation}
\lim_{q/M_{W} \to 0} \frac{g^{2}}{8(q^{2} - M_{W}^{2})} = \frac{G_{F}}{\sqrt{2}}.
\end{equation}
The weakness of the interactions when compared to QED can be derived from the fact that unlike the photon the propogators in the weak interaction are massive.
Like QED in Section \ref{sec:qed}, the weak interaction is also a gauge symmetry. 
Unlike QED however, the group is $SU(2)_{L}$, the $L$ representing that left handed states are $SU(2)$ doublets, and right handed states are singlets.
The $SU(2)$ group yields the three gauge bosons, $W^{\pm}$ and $Z^{0}$, of the weak force.
There is however a problem in that adding massive gauge bosons breaks gauge invariance and leads to a non-renormalizable theory.
This problem will be discussed in Sections \ref{sec:breaking} and \ref{sec:electroweak}.
\begin{figure}[htpb]
\begin{center}
\begin{fmffile}{betadecay} 	%one.mf will be created for this feynman diagram  
  \fmfframe(1,7)(1,7){ 	%Sets dimension of Diagram
   \begin{fmfgraph*}(110,62) %Sets size of Diagram
    \fmfleft{i1}	%Sets there to be 2 sources 
    \fmfright{i2,o1,o2}    %Sets there to be 2  outputs
    \fmflabel{$d$}{i1} %Labels one of the left sources
    \fmflabel{$u$}{i2} %Labels one of the left sources
    \fmflabel{$\overline{\nu}_{e}$}{o1} %Labels one of the right outputs
    \fmflabel{$e^{-}$}{o2} %Labels one of the right outputs
    \fmf{fermion}{i1,v1,i2} %Connects the sources with a vertex.
    \fmf{fermion}{o1,v2,o2} %Connects the outputs with a vertex.
    \fmf{photon,label=$W^{-}$}{v1,v2} %Labels the conneting line.
   \end{fmfgraph*}
  }
\end{fmffile}
\end{center}
\caption{Feynman diagram of nuclear beta decay via a massive intermediate gauge boson $W^{-}$.}
\label{fig:betadecay}
\end{figure}

