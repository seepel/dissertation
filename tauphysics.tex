\section{The Tau Lepton}
The Tau lepton ($\tau$) is the heaviest of the three leptons with a mass of $1.87$ GeV. 
It also has the shortest lifetime, traveling a typical distance of $86$ $\mu$m before decaying.
This means that the tau lepton will decay before reaching any active detector element. 
As will be discussed in Section \ref{sec:tracker}, the charged particle tracking system used at CMS has very good spatial resolution and can reconstruct the displaced decay vertex of a tau lepton.
The tau lepton can decay leptonically to either an electron or a muon, along with two neutrinos.
More likely  however is that the tau will decay hadronically, producing a low multiplicity and collimated jet of mesons. 
The jet resulting from a tau lepton decay typically contains $\pi^{\pm}$ and $\pi^{0}$ mesons, along with a single neutrino which at high energies will be colinear with the jet.
When decaying hadronically the tau lepton often passes through an intermediate resonance (the $\rho$ meson for example), which can be seen in Table \ref{tab:taudecays} along with the branching ratios for the most common decay modes.

The properties of the tau lepton make it a difficult physics object to deal with at hadron colliders, for example when compared to the muon.
The first difficulty arises in that the tau decay always has an associated neutrino which interacts only weakly and cannot be directly detected.
The tau lepton decays hadronically approximately 65\% of the time which leads to the second difficulty. 
Hadron colliders typically have large QCD multi-jet activity, thus identifying a hadronic tau decay against other jets in an event can be difficult.
In order to accomplish this, the low multiplicity of a hadronic tau decay is leveraged. 
Further discrimination power can be achieved by checking for the compatibility of an intermediate resonance.
One can consider the implications of these difficulties in terms of the different final states that the $H\rightarrow\tau\tau$ channel will have.
The case where both taus decay leptonically would yield a much cleaner signal, but the branching ratio for this case is very small.
Additionally, in the case where both taus decay to the same flavor of lepton, the signal would be very difficult to distinguish from $Z\rightarrow ll$ events.
In contrast the branching ratio is significantly higher in the case that both taus decay hadronically, this however leads to very high QCD multi-jet backgrounds which will make the analysis more difficult.
One can find a satisfactory middle ground in the case where one tau decays leptonically and the other decays hadronically.
If the leptonic leg of the decay is an electron, the backgrounds from $Z\rightarrow ee$ are found to be a bit higher than the case of a muon. 
The higher backgrounds from $Z\rightarrow ee$ events is due to electrons having a higher mis-identification rate than muons.
Thus the search for the Higgs boson decaying to two tau leptons in which one tau decays to a muon and the other decays hadronically is a favored channel in that it optimizes the production rate and branching ratio while minimizing the backgrounds of the process.

\begin{table}[htpb]
  \begin{center}
    \caption{TAU LEPTON DECAY MODES}
    \label{tab:taudecays}
    \begin{tabular}{lccr}
      \toprule
      Final State & Resonance & Mass (MeV) & Branching Ratio \\
      \midrule
%      $e^{-}\nu_{\tau}\overline{\nu}_{e}$ & - & $0.5$ & $17.8\%$ \\
%      $\mu^{-}\nu_{\tau}\overline{\nu}_{\mu}$ & - & $105$ & $17.4\%$ \\
%      $\pi^{-}\nu_{\tau}$ & - & $135$ & $10.9\%$ \\
      $e^{-}\nu_{\tau}\overline{\nu}_{e}$      & -      & -      & $17.8\%$ \\
      $\mu^{-}\nu_{\tau}\overline{\nu}_{\mu}$  & -      & -      & $17.4\%$ \\
      $\pi^{-}\nu_{\tau}$                      & -      & -      & $10.9\%$ \\
      $\pi^{-}\pi^{0}\nu_{\tau}$               & $\rho$ & $770$  & $25.5\%$ \\
      $\pi^{-}\pi^{0}\pi^{0}\nu_{\tau}$        & $a1$   & $1260$ & $9.3\%$  \\
      $\pi^{-}\pi^{-}\pi^{+}\nu_{\tau}$        & $a1$   & $1260$ & $9.0\%$  \\
      $\pi^{-}\pi^{-}\pi^{+}\pi^{0}\nu_{\tau}$ & $a1$   & $1260$ & $4.5\%$  \\
      \bottomrule
    \end{tabular}
  \end{center}
\end{table}

