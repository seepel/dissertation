\section{Systematic Uncertainties}
\label{sec:systematics}
\subsection{Normalization Uncertainties}
Each of the efficiency correction factors calculated in Sections \ref{subsec:triggercorrections} and \ref{subsec:leptoncorrections} translates directly into an uncertainty on the final event yield.
This uncertainty is applied as an overall normalization uncertainty determined from the uncertainties in the given measurements.
The combined uncertainty on corrections applied in relation to tau leptons is estimated to be $6.0\%$, and $2.0\%$ for corrections in relation to muons.

The jet energy scale and uncertainty is measured by means of jet balancing in di-jet and $Z/\gamma$+jet events\cite{JETENERGYSCALE}.
An uncertainty of $2.0\% - 5.0\%$ is taken for the true jet energy scale as a function of $p_{T}$ and $\eta$.
The uncertainty on the scale of the jet energy scale is applied as an uncertainty on the yield after being extrapolated to the signal region via a weighted sum given the $p_{T}$ and $\eta$ of the jets in each category.

The uncertainty on the total integrated luminosity will enter the analysis by means of an uncertainty on the normalization of the signal simulation and in any background simulations that have been scaled by the luminosity, e.g. $Z\rightarrow\tau\tau$.
An uncertainty on the luminosity measurement of $4.5\%$ is applied as an uncertainty on the normalization of all signal samples and background samples for which the normalization is taken from simulation\cite{LUMI}. The uncertainty on the $Z \rightarrow \tau\tau$ normalization is taken as $2.5\%$ from the inclusive cross section measurement made by the CMS Collaboration\cite{Z_TAUTAU_CROSS_SECTION}. 
However, rather than taking the uncertainty on the integrated luminosity as an uncorrelated uncertainty, a reduced uncertainty of $2.0\%$ is taken as the uncorrelated portion of the uncertainty
The combined normalization uncertainty on the $Z\rightarrow\tau\tau$ background is then $3.2\%$. 

The normalization of the backgrounds arising from $Z \rightarrow \mu\tau_{jet}$ events results in an uncertainty between $20\% - 25\%$ derived by the uncertainty in extrapolating from the control region into the signal region.
Likewise the uncertainty on the background related to the process $Z\rightarrow\mu\tau_{\mu}$ is between $0.01\% - 0.04\%$, which is driven down by the large number of $Z\rightarrow\mu\mu$ events in the control region for this process.
For the categories \emph{Zero/One Jet}, \emph{B-Tagged} and \emph{Non B-Tagged}, the extrapolation method for estimating the $t\overline{t}$ background produces an uncertainty of $4.1\%$, $2.5\%$, and $4.1\%$ respectively, while for the \emph{Boost} and \emph{VBF} the normalization is taken from the CMS cross section measurement similar to the procedure followed for the $Z\rightarrow\tau\tau$ background and is estimated to be $7.5\%$.
In a similar fashion the uncertainty on the normalization of the $W$ background contribution is obtained from the uncertainty in extrapolating from the control region into the signal region yielding a value of $7\%$ across all categories. 

\subsection{Shape Uncertainties}
In addition to pure normalization uncertainties related to the rate of each process, there are additional uncertainties that will modify the shape of the $m_{\tau\tau}$ distribution for a given process as taken from the simulation.
The variables that contribute to the uncertainty on the distribution are the $p_{T}$ of the muon, the energy of the tau, and the $\met$.
The source of these uncertainties is ultimately due to inaccuracies in the simulation. 
Rather than correct these inaccuracies, however, they are taken into account via a shape uncertainty.
The shape uncertainty is accounted for by varying the variable of interest up and down by one standard deviation, producing the scenarios in which the variable is shifted up and down.

\subsection{Theory Uncertainties}
In addition to the previously discussed uncertainties additional uncertainties related to theoretical calculations must be taken into account.
The first theoretical uncertainty to be considered is the uncertainty of the parton-distribution functions (PDFs) that describe the distribution of momentum shared among the constituents of the colliding protons.
This uncertainty is taken to be $3.0\%$ and is derived from \cite{PDFUNCERTAINTY}.

In the standard model search the uncertainty on the dependence of the signal acceptance on the Higgs production mechanism and branching ratio are taken directly into account in the cross section measurement.
This includes the uncertainty on the gluon-gluon fusion and vector-boson fusion production mechanisms ($gg \rightarrow H$ and $qq \rightarrow Hqq$ respectively)\cite{HIGGS_PRODUCTION}.
In the MSSM search the production mechanism uncertainty is used when translating the cross section measurement to a search in $M_{A}-tan\beta$ space.
