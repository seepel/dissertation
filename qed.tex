\subsection{Quantum Electrodynamics}
\label{sec:qed}
QED is an Abelian gauge theory, meaning that it is invariant under local gauge transformations.
The ``gauge field'' of the theory describes the interaction of particles in accordance with the electromagnetic force.
QED has seen great success in predicting experimental results, and has in fact done so with phenomenal precision.

The Dirac equation is a logical starting point for the explanation of QED. 
It is a wave equation describing 1/2 spin particles.
By incorporating the transformation,
\begin{equation}
\label{eqn:quantumderivative}
p_{\mu} \rightarrow i\partial_{\mu},
\end{equation}
into the special relativistic energy/momentum relationship,
\begin{equation}
p^{\mu}p_{\mu} - m^{2} = 0,
\end{equation}
one can indeed derive a quantum mechanical equation compatible with special relativity, and this is the approach taken in the Klein-Gordon equation.
Dirac, however, sought to solve this problem by deriving an equation that had a first order derivative in time, which led to:
\begin{equation}
\label{eqn:diracspecial}
(\gamma^{\kappa}p_{\kappa} + m)(\gamma^{\mu}p_{\mu} -m) = 0.
\end{equation}
By substituting Equation \ref{eqn:quantumderivative} into one of the terms in Equation \ref{eqn:diracspecial}, one obtains the equation of motion for a spin 1/2 particle:
\begin{equation}
\label{eqn:diracmotion}
i\gamma^{\mu}\partial_{\mu}\psi - m\psi = 0.
\end{equation}
In Equations \ref{eqn:diracspecial} and \ref{eqn:diracmotion}, $\gamma^{\mu}$ represents the set of four Dirac matrices specified using the Pauli matrices,
%\begin{equation}
%\begin{array}{rl}
%\gamma^{0} = \begin{pmatrix} 1 & 0 & 0 & 0 \\ 0 & 1 & 0 & 0 \\ 0 & 0 & 1 &0 \\ 0 & 0 & 0 & 1\end{pmatrix}, & \gamma^{1} = \begin{pmatrix} 0 & 0 & 0 & 1 \\ 0 & 0 & 1 & 0 \\ 0 & -1 & 0 & 0 \\ -1 & 0 & 0 & 0\end{pmatrix}, \\
%\gamma^{2} = \begin{pmatrix} 0 & 0 & 0 & -i \\ 0 & 0 & i & 0 \\ 0 & i & 0 & 0 \\ -i & 0 & 0 & 0\end{pmatrix}, & \gamma^{3} = \begin{pmatrix} 0 & 0 & 1 & 0 \\ 0 & 0 & 0 & -1 \\ -1 & 0 & 0 & 0 \\ 0 & 1 & 0 & 0\end{pmatrix}, \\
%\end{array}
%\end{equation}
and the solution $\psi$ is %a 1x4 (FIXME?)matrix 
referred to as a Dirac spinor.

The Lagrangian ($\Lagr$), resulting from the Dirac equation is
\begin{equation}
\label{eqn:globallagrangian}
\Lagr = \overline{\psi}(i \gamma^{\mu}\partial_{\mu} - m)\psi
\end{equation}
and is invariant under the global gauge transformation
\begin{equation}
\label{eqn:globalinvariance}
\psi^{\prime}  \rightarrow e^{i\theta}\psi,
\end{equation}
indicating that the global symmetry group is $U(1)$.
Recall, however, that QED must be invariant under local gauge transformations.
This would require the Lagrangian to be invariant under Equation \ref{eqn:globalinvariance} in the case that $\theta$ is a function of position ($\theta = \theta(x)$).
Although $\partial_{\mu}$ commutes with a constant $\theta$ it does not commute with $\theta(x)$ leading to the following modification:
\begin{equation}
\partial_{\mu} \rightarrow D_{\mu} = \partial_{\mu} - ieA^{\mu}.
\end{equation}
This adds an additional term to the Lagrangian (Equation \ref{eqn:globallagrangian}) of the form $e\psibar\gamma^{\mu}\psi A_{\mu}$, which can be interpreted as the coupling of the gauge field with the fermion.
The strength of this coupling is proportional to $e$, the electric charge, which recovers the observation that neutral particles do not interact electromagnetically.
Additionally, this term has the added interpretation that the electromagnetic force between two charged particles is caused by the exchange of a gauge boson, in this case a photon.
This does not complete the picture however, as there is no kinetic term for the photon. 
Without a kinetic term, there is no room in the theory for photons to traverse space.
The full QED Lagrangian is accomplished by adding this kinetic term using the field strength tensor:
\begin{equation}
\label{eqn:qedlagrangian}
\Lagr = \psibar (i\gamma^{\mu}D_{\mu} - m)\psi - \frac{1}{4}F_{\mu\nu}F^{\mu\nu}.
\end{equation}

QED describes the electromagnetic force via the U(1) gauge symmetry, and successfully merges quantum mechanics with special relativity.
The gauge group that QED belongs to dictates the behavior of the gauge boson, and how that boson interacts with other particles.
