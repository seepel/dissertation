\section{Beyond the Standard Model}
\label{sec:bsd}

The SM has seen tremendous success in its predictive power, and indeed there have been very few experiments that are incompatible with the SM.
There are, however, discrepancies that lead one to believe that the SM is not complete.
The SM can be thought to be an effective theory that is valid at energies below some cut off scale ($\Lambda$).
The challenge is then to extend the SM to include new physics above $\Lambda$ while maintaining consistency with the SM below $\Lambda$.

One obvious omission in the SM is the inclusion of gravity. 
To date there is still no verifiable model for which gravity and quantum mechanics are compatible.
Beyond gravity there are still other reasons to believe new physics must lie beyond the SM.
Cosmological observations indicate that the universe has an additional massive amount of matter that interacts only weakly, commonly referred to as dark matter.
More relevant to this analysis, and the Higgs boson, there is the hierarchy problem.
The hierarchy problem presents itself in numerous ways. 
Here the leading order correction to the Higgs mass is considered.
The one loop fermion correction to the Higgs mass can be seen in the left side of Figure \ref{fig:higgsfermion}. 
This diagram contributes approximately $\Delta m_{H}^{2} \approx \Lambda^{2}$.
If one assumes $\Lambda \approx M_{planck} \approx \mathcal{O}(10^{19})$, there then needs to exist higher order corrections $\mathcal{O}(10^{38})$. 
In other words the higher order corrections need to be 36 orders of magnitude larger than the Higgs mass itself.
Higher order corrections of such magnitude seem very unnatural.
One may therefore look to new physics to bound the calculation of the Higgs mass.
\begin{figure}
\begin{minipage}{0.5\linewidth}
\begin{center}
\begin{fmffile}{higgsfermion} 	%one.mf will be created for this feynman diagram  
\fmfframe(1,7)(1,7){ 	%Sets dimension of Diagram
\begin{fmfgraph*}(110,62) %Sets size of Diagram
\fmfleft{i}
\fmfright{o}
\fmflabel{$H$}{i}
\fmflabel{$H$}{o}
\fmf{dashes}{i,v1}
\fmf{dashes}{v2,o}
\fmf{fermion,left,tension=.4}{v1,v2,v1}
\end{fmfgraph*}
}
\end{fmffile}
\end{center}
\end{minipage}
%\hspace{0.1cm}
\begin{minipage}{0.5\linewidth}
\begin{center}
\begin{fmffile}{higgsscaler} 	%one.mf will be created for this feynman diagram  
\fmfframe(1,7)(1,7){ 	%Sets dimension of Diagram
\begin{fmfgraph*}(110,62) %Sets size of Diagram
\fmfbottom{i,o}
\fmflabel{$H$}{i}
\fmflabel{$H$}{o}
\fmf{dashes}{i,v}
\fmf{dashes}{v,o}
\fmf{dashes,right,tension=.7}{v,v}
\end{fmfgraph*}
}
\end{fmffile}
\end{center}
\end{minipage}
\caption{One loop fermion (left) and sparticle (right) corrections to the Higgs boson mass.}
\label{fig:higgsfermion}
\end{figure}

One of the arguably more appealing theories to extend the SM is supersymmetry, which is aimed at solving the hierarchy problem.
Supersymmetry extends the SM by introducing an additional symmetry between bosons and fermions.
This symmetry has the effect of adding new particles to the SM in such a way that each additional particle is paired with a SM particle. 
In each pairing, all of the quantum numbers remain the same except that the spin of the partner particle is modified by 1/2.
Each fermion has a super partner that is a scalar boson, referred to by prepending an ``s'', for scalar, to the name of the SM particle, giving the collection of sparticles.
Each gauge boson has a fermion super partner, the nomenclature here is to append ``ino'' to the name of the particle, for example gluino, collectively referred to as gauginos.
This effectively solves the hierarchy problem in that for each fermion loop diagram , there is an accompanying diagram that enters in with opposite sign for the super partner as shown in the right side of Figure \ref{fig:higgsfermion}.
In most models of supersymmetry the lightest of the supersymmetric particles (LSP) is necessarily stable. 
This leads to an additional benefit in the form of an attractive dark matter candidate.

In order to construct the algebra for supersymmetry groups, each SM particle and its supersymmetric partner must be grouped into either chiral or gauge supermultiplets. 
A chiral supermultiplet contains a left handed or right handed fermion and a complex scalar, while a gauge supermultiplet contains a spin-1 vector boson and a Majorana gaugino fermion.
It is then clear that the Higgs must exist in a chiral supermultiplet, as a corollary to this requirement there must be at least one additional Higgs supermultiplet.
The requirement of a second Higgs supermultiplet is ultimately due to gauge anomalies that are introduced with the Higgsino which must be canceled by a second supermultiplet of opposite hypercharge.
To summarize, in order to have a supersymmetric extension to the SM one must double the particle content of the SM by creating supermultiplets and add at least one additional Higgs supermultiplet. 
This is known as the Minimal Supersymmetric Standard Model (MSSM).

\subsection{Minimal Supersymmetric Standard Model}
\label{sec:mssm}
The content of the MSSM has been set forth at the beginning of this section and is summarized in Tables \ref{tab:chiralsuper} and \ref{tab:gaugesuper}. 
Here the phenomenological repercussions of the MSSM are briefly discussed, while a full treatment of supersymmetry can be found in \cite{SUSY}.
The superpotential of the MSSM is:
\begin{equation}
W_{MSSM} = \overline{u}\mathbf{y_{u}}QH_{u} - \overline{d}\mathbf{y_{d}}QH_{d} - \overline{e}\mathbf{y_{e}}LH_{d} + \mu H_{u}H_{d}.
\end{equation}
where $\mathbf{y}$ are the Yukawa 3 x 3 matrices.
Here one can note that the up type fermions couple to the $H_{u}$ doublet while the down type fermions to the $H_{d}$ doublet.
The scalar potential ($V$) at the minimum is found to be similar to that in the SM:
\begin{equation}
\begin{array}{rcl}
\label{eqn:superpotential}
V & = & (|\mu|^{2} + m_{H_{u}}^{2})|H_{u}^{0}|^{2} + (|\mu|^{2} + m_{H_{d}}^{2})|H_{d}^{0}|^{2} \\
  & - & (bH_{u}^{0}H_{d}^{0} + c.c.) + \frac{1}{8}(g^{2} + g^{\prime 2})(|H_{u}^{0}|^{2} - |H_{d}^{0}|^2)^{2}.
\end{array}
\end{equation}
In Equation \ref{eqn:superpotential} both $H_{u}$ and $H_{d}$ can acquire a VEV, $\nu_{u}$ and $\nu_{d}$ related to the VEV in the SM via:
\begin{equation}
\nu_{u}^{2} + \nu_{d}^{2} = \nu^{2} = \frac{2M_{Z}^{2}}{g^{2} + g^{\prime 2}} \approx (175 GeV)^{2}.
\end{equation} 
In the MSSM, the ratio of the VEVs is commonly referred to as the parameter $tan\beta$ as shown in:
\begin{equation}
tan\beta \equiv \frac{\nu_{u}}{\nu_{d}},
\end{equation}
which has significant phenomenological consequences.
Most notably, for the purposes of this analysis, the Yukawa couplings of the SM fermions are dependent upon $tan\beta$ as follows:
\begin{equation}
\begin{array}{rcl}
\label{eqn:mssmcouplings}
m_{t} & = & \mathbf{y_{t}}\nu sin\beta, \\
m_{b} & = & \mathbf{y_{b}}\nu cos\beta, \\
m_{\tau} & = & \mathbf{y_{\tau}}\nu cos\beta.
\end{array}
\end{equation}
When $tan\beta$ becomes large, the couplings for the $b$ and $\tau$ become enhanced. 
In the MSSM, the Higgs sector has eight degrees of freedom compared to the four degrees of freedom in the SM. 
As in the SM three of those degrees of freedom give mass to the $W^{\pm}$ and $Z$ bosons, while the remaining five create five massive Higgs bosons.
The five Higgs bosons in the MSSM are two CP-even neutral scalars, $h^{0}$ and $H^{0}$, one CP-odd neutral scalar, $A^{0}$, and two charged scalars, $H^{\pm}$.
At tree level the mass of the three neutral Higgs bosons are related to each other along with $tan\beta$ by the equations
\begin{equation}
\begin{array}{rcl}
\label{eqn:higgsmasses}
m_{h^{0}}^{2} & = & \frac{1}{2}\left(m_{A^{0}}^{2} + m_{Z}^{2} - \sqrt{\left(m_{A^{0}}^{2} - m_{Z}^{2}\right)^{2} + 4m_{A^{0}}^{2}m_{Z}^{2}sin^{2}(2\beta)}\right), \\
m_{H^{0}}^{2} & = & \frac{1}{2}\left(m_{A^{0}}^{2} + m_{Z}^{2} + \sqrt{\left(m_{A^{0}}^{2} - m_{Z}^{2}\right)^{2} + 4m_{A^{0}}^{2}m_{Z}^{2}sin^{2}(2\beta)}\right). \\
\end{array}
\end{equation}
In Equation \ref{eqn:higgsmasses}, one can note a few interesting behaviors, first the mass of $m_{h^{0}}$ is bounded from above by $m_{Z}$.
Secondly, $m_{A^{0}}$ has the artifact of being nearly degenerate with either $m_{h^{0}}$ at low mass or $m_{H^{0}}$ at high mass.

\begin{table}[htpb]
  \begin{center}
    \caption{CHIRAL SUPERMULTIPLETS IN THE MSSM}
    \label{tab:chiralsuper}
    \begin{tabular}{lccccc}
      \toprule
      Names & symbol & spin 0 & spin 1/2 & $SU(3)_{C}, SU(2)_{L}, U(1)_{Y}$ \\
      \midrule
      squarks, quarks & $Q$ & $\begin{pmatrix}\tilde{u}_{L} & \tilde{d}_{L}\end{pmatrix}$ & $\begin{pmatrix}u_{L} & d_{L}\end{pmatrix}$ & $\begin{pmatrix}3, & 2, & \frac{1}{6}\end{pmatrix}$ \\
                      & $\overline{u}$ & $\tilde{u}_{R}^{*}$  & $u_{R}^{\dagger}$ & $\begin{pmatrix}\overline{3}, & 1, & -\frac{2}{3}\end{pmatrix}$ \\
                      & $\overline{d}$ & $\tilde{d}_{R}^{*}$  & $d_{R}^{\dagger}$ & $\begin{pmatrix}\overline{3}, & 1, & \frac{1}{3}\end{pmatrix}$ \\
      \midrule
      sleptons, leptons & $L$ & $\begin{pmatrix}\tilde{\nu} & \tilde{e}_{L}\end{pmatrix}$ & $\begin{pmatrix}\nu & e_{L}\end{pmatrix}$ & $\begin{pmatrix}1, & 2, & -\frac{1}{2}\end{pmatrix}$ \\
                      & $\overline{e}$ & $\tilde{e}_{R}^{*}$  & $e_{R}^{\dagger}$ & $\begin{pmatrix}1, & 1, & 1\end{pmatrix}$ \\
      \midrule
      Higgs, higgsinos & $H_{u}$ & $\begin{pmatrix}H_{u}^{+} & H_{u}^{0}\end{pmatrix}$ & $\begin{pmatrix}\tilde{H}_{u}^{+} & \tilde{H}_{u}^{0}\end{pmatrix}$ & $\begin{pmatrix}1, & 2, & +\frac{1}{2}\end{pmatrix}$ \\
                       & $H_{d}$ & $\begin{pmatrix}H_{d}^{+} & H_{d}^{0}\end{pmatrix}$ & $\begin{pmatrix}\tilde{H}_{d}^{+} & \tilde{H}_{d}^{0}\end{pmatrix}$ & $\begin{pmatrix}1, & 2, & -\frac{1}{2}\end{pmatrix}$ \\
      \bottomrule
    \end{tabular}
  \end{center}
\end{table}
\begin{table}[htpb]
  \begin{center}
    \caption{GAUGE SUPERMULTIPLETS IN THE MSSM}
    \label{tab:gaugesuper}
    \begin{tabular}{lccccc}
      \toprule
      Names & spin 1/2 & spin 1 & $SU(3)_{C}, SU(2)_{L}, U(1)_{Y}$ \\
      \midrule
      gluino, gluon & $\tilde{g}$ & $g$ & $\begin{pmatrix}8, & 1, & 0\end{pmatrix}$ \\
      winos, W bosons & $\tilde{W}^{\pm}$ $\tilde{W}^{0}$ & $W^{\pm}$ $W^{0}$ & $\begin{pmatrix}1, & 3, & 0\end{pmatrix}$ \\
      binos, B boson & $\tilde{B}^{0}$ & $B^{0}$ & $\begin{pmatrix}1, & 1, & 0\end{pmatrix}$ \\
      \bottomrule
    \end{tabular}
  \end{center}
\end{table}

