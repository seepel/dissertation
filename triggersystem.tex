\section{Trigger System}
\label{sec:triggersystem}
%At the nominal LHC luminosity of $10^{34}$ \lumiunit, bunch crossings will occur every 25 ns, which would correspond to an interaction rate of approximately $40$ MHz.
At the LHC design specifications, bunch crossings will occur every 25 ns. 
This rate would correspond to an interaction rate of approximately $40$ MHz.
Considering the complexity of the CMS detector, the bandwidth required to read and store data from every bunch crossing would be greater that one Petabit per second.
In addition to the unrealistic requirements of storing every event, is the fact that the cross section for common QCD interactions is many orders of magnitude larger than that of ``interesting'' physics.
These reasons lead to the obvious solution that there must be a system by which interesting events are kept, and common events are discarded.
The CMS trigger system is designed with this task in mind and consists of a very fast first-level trigger (L1), and a  more complete high level trigger (HLT).

The designed acceptance rate of the L1 trigger is $100$ kHz. 
In order to achieve this the L1 trigger is comprised of custom electronics that are built into the subdetector subsystems.
The latency of the L1 trigger is restricted to $3.2$ \microunit{s} and as such only information from the calorimeters and muon system are available, as the time constraint on reconstructing tracks is too great.
Each subdetector system generates a set of trigger primitives, typically photon/electron objects, muons and jets with certain $E_{T}$ or $p_{T}$ thresholds.
The L1 trigger also integrates global requirements such as sums over transverse energy ($E_{T}$) and missing transverse energy ($\met$) from the calorimeters.
After a successful L1 trigger decision, the event is fed into memory pipelines for read out by the Data Acquisition System (DAQ) and ultimately the HLT.
Unlike the L1 trigger which relies on fast dedicated hardware, the HLT is run in software on a farm of commercial computers.
The acceptance rate of the HLT is required to be on the order of $100$ Hz giving room for more complex operations such as reconstructing tracks, but the time to analyze each collision remains tight.
The HLT is made up of several ``paths'' that typically represent more complex versions/combinations of the L1 trigger primitives, such as a single photon with a given $E_{T}$ threshold or a muon and tau in coincidence.
Running the HLT in software allows for a greater flexibility to meet changing demands based on requirements such as increasing luminosity, or even hints towards new physics.

This analysis uses a cross trigger which requires both a muon and a hadronically decaying tau to be present in order to pass the HLT.
The triggers used are all based upon a single muon L1 seed, then in the HLT path both a muon and a tau object are required to be present.
In order to reduce rates as the luminosity increased over the 2011 running period, several different triggers were used throughout different run ranges. 
The triggers used will be described in more detail in Chapter \ref{chap:analysis}.

